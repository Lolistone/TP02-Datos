\documentclass[10pt,a4paper]{article}

\input{AEDmacros}
\usepackage{caratula} % Version modificada para usar las macros de algo1 de ~> https://github.com/bcardiff/dc-tex


\titulo{Trabajo Práctico 2}
\subtitulo{Clasificación y Selección de Modelos}

\fecha{\today}

\materia{Laboratorio de Datos}
\grupo{Grupo 100}

\integrante{Apellido, Nombre1}{001/01}{email1@dominio.com}
\integrante{Martinelli, Lorenzo}{364/23}{martinelli.lorenzo12@gmail.com}
\integrante{Padilla, Ramiro}{1636/21}{ramiromdq123@gmail.com}
% Pongan cuantos integrantes quieran

% Declaramos donde van a estar las figuras
% No es obligatorio, pero suele ser comodo
\graphicspath{{../static/}}

\begin{document}

\maketitle

\section{Introducción}

\subsection{Fuente de datos}

A lo largo de este proyecto, trabajaremos con un conjunto de datos de imágenes denominado Sign
Language MNIST, el cual se encuentra en formato csv, donde cada imagen del set de datos representa una letra en lenguaje de
señas americano. El link al dataset es el siguiente https://www.kaggle.com/datasets/datamunge/sign-language-mnist.

\subsection{Análisis exploratorio de datos}

\section{Modelo}
Realizando distintas pruebas, notamos que utilizando distintas cantidades de pixeles, a medida que el numero de vecinos incrementa
la performance decae, siendo el número ideal de vecinos exactamente 1.

\begin{figure}[ht!]
	\begin{subfigure}{0.5\textwidth}
		\includegraphics[width=0.9\linewidth]{Informe/Imagenes/50pixeles.png} 
		\caption{Los 3 pixeles con mayor varianza}
		\label{fig:subfig1}
	\end{subfigure}
	\begin{subfigure}{0.5\textwidth}
		\includegraphics[width=0.9\linewidth]{Informe/Imagenes/3pixeles.png}
		\caption{50 pixeles en una zona de varianza media}
		\label{fig:subfig2}
	\end{subfigure}
	% OJO: el caption siempre va antes del label
	\label{fig:subfigs}
\end{figure}

Sin embargo, encontramos que cuando tenemos solo un pixel. Sin importar si este es significativo o no, la performance del modelo tiende a aumentar a medida
que el numero de vecinos es mayor. En este caso, a partir de los gráficos, pareciera ser que k = 9 sería un buen número.

\begin{figure}[ht!]
	\begin{subfigure}{0.5\textwidth}
		\includegraphics[width=0.9\linewidth]{Informe/Imagenes/1pixelalto.png} 
		\caption{Pixeles de alta varianza}
		\label{fig:subfig1}
	\end{subfigure}
	\begin{subfigure}{0.5\textwidth}
		\includegraphics[width=0.9\linewidth]{Informe/Imagenes/1pixelbajo.png}
		\caption{Pixeles de baja varianza}
		\label{fig:subfig2}
	\end{subfigure}
	% OJO: el caption siempre va antes del label
	\label{fig:subfigs}
\end{figure}








\end{document}
