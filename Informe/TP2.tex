\documentclass[10pt,a4paper]{article}

\input{AEDmacros}
\usepackage{caratula} % Version modificada para usar las macros de algo1 de ~> https://github.com/bcardiff/dc-tex


\titulo{Trabajo Práctico 2}
\subtitulo{Clasificación y Selección de Modelos}

\fecha{\today}

\materia{Laboratorio de Datos}
\grupo{Grupo 100}

\integrante{Apellido, Nombre1}{001/01}{email1@dominio.com}
\integrante{Martinelli, Lorenzo}{364/23}{martinelli.lorenzo12@gmail.com}
\integrante{Apellido, Nombre3}{003/01}{email3@dominio.com}
% Pongan cuantos integrantes quieran

% Declaramos donde van a estar las figuras
% No es obligatorio, pero suele ser comodo
\graphicspath{{../static/}}

\begin{document}

\maketitle

\section{Introducción}

\subsection{Fuente de datos}

A lo largo de este proyecto, trabajaremos con un conjunto de datos de imágenes denominado Sign
Language MNIST, el cual se encuentra en formato csv, donde cada imagen del set de datos representa una letra en lenguaje de
señas americano. El link al dataset es el siguiente https://www.kaggle.com/datasets/datamunge/sign-language-mnist.

\subsection{Análisis exploratorio de datos}

(Acá basicamente narramos lo que hicimos en la consigna 1)

deje este fragmento de codigo porque lo quierp usar
\begin{figure}[ht!]
	\begin{subfigure}{0.5\textwidth}
		\includegraphics[width=0.9\linewidth]{1pixelalto.png} 
		\caption{Pixeles de alta varianza}
		\label{fig:subfig1}
	\end{subfigure}
	\begin{subfigure}{0.5\textwidth}
		\includegraphics[width=0.9\linewidth]{1pixelbajo.png}
		\caption{Pixeles de baja varianza}
		\label{fig:subfig2}
	\end{subfigure}
	% OJO: el caption siempre va antes del label
	\label{fig:subfigs}
\end{figure}








\end{document}
